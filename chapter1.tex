\chapter{Divisors and Invertible Sheaves}
\label{chapter_divisors and invertible sheaves}

\onehalfspacing

\section{Weil Divisors}
In this section, we will define what divisors are, which are an important tool to study geometrical properties of objects such as varieties and schemes. We will also define linear equivalence of divisors, and divisor class group. Divisor class groups are invariants for schemes and varieties, and hence an very important property for classification of many varieties. We can also use this for projective spaces, which we would require in the proof of Riemann-Roch theorem, the goal of this report.

There are mainly two different ways to define divisors: \textit{Weil} and \text{Cartier}. The former has a simpler geometric interpretation, but can only be defined for certain Noetherian integral schemes. Cartier divisors on the other hand are defined for general schemes. There is also a relation between Weil and Cartier divisors and invertible sheaves, which we treat at the end of this section. \par

\begin{example}
    \label{example_1_1_1}
   Let $C$ be a non-singular projective curve in $\bb{P}^2_{k}$. For each line $L$ in $\bb{P}^2_{k}$, we consider $L \cap C$. Hence, $L \cap C$ is a finite set of points on C.

Let $C$ be of degree $d$. Then, counting multiplicities, $L \cap C$ contains exactly $d$ many points. We now denote $L \cap C$ as the following:
$$L \cap C = \sum n_iP_i$$ where $P_i$ are the points in $L \cap C$, and $n_i$ is the multiplicity of $P_i$. This formal sum is called a \textit{\textcolor{ProcessBlue}{divisor over $C$}}.  

Now we vary $L$. Hence, we get a family of divisors, parametrized by each line $L$ in $\bb{P}^2_{k}$. The set of lines $L$ in $\bb{P}^2_{k}$ is also the dual projective space $(\bb{P}^2_{k})^*$, hence the family of divisors is parametrized by the points of $(\bb{P}^2_{k})^*$. We call this set of divisors a \textit{\textcolor{ProcessBlue}{linear system on $C$}}. 
\end{example}
\begin{remark}
\label{remark_1_1_1}
Note that if $P$ is a point in $C$, it lies in some $L \cap C$ for some line $L$ in $\bb{P}^2_{k}$. We take all such divisors containing $P$. Then the divisors correspond to the set of lines in $\bb{P}^2_{k}$ which pass through $P$, and hence uniquely determine $P$ in $\bb{P}^2_{k}$. So, we can recover $C$ as an embedding in $\bb{P}^2_{k}$ if we know of this linear system. 
\end{remark}

\begin{definition}
    A scheme $X$ is said to be \textit{\textcolor{ProcessBlue}{regular (or non-singular) in codimension one}} if every local ring $\cc{O}_{X,x}$ of $X$ of dimension one is regular.
\end{definition}

The most important examples of such schemes are non-singular varieties over a field, and Noetherian normal schemes. On a non-singular variety over a field, every local ring of a closed point is regular. So, all local rings being localizations of the local rings of the closed points, are regular. For Noetherian normal schemes, every local ring of dimension one are integrally closed, hence regular.

For our ease of access, we say that $X$ satisfies the property $(*)$ if $X$ is a Noetherian integral seperated scheme which is regular in codimension one. We'd be referring to this whenever it seems repetitive to mention the property over and over.

\begin{remark}
    \label{remark_1_1_2}
    We know from (\cite{one}, II, Ex. 3.6)  that if $X$ is a scheme and $Y \subseteq X$ a closed irreducible subset with generic point $\eta$, then $\dim \mathcal{O}_{X, \eta} = \operatorname{codim}(Y, X)$.
\end{remark}

\begin{remark}
\label{remark_1_1_3}
    If $X$ is an integral scheme with generic point $\xi$ then $K = \mathcal{O}_{X, \xi}$ is a field and for every $x \in X$ there is an injection of rings $\mathcal{O}_{X, x} \longrightarrow K$ such that $K$ is the quotient field of all these local rings. Consequently if two sections $s \in \mathcal{O}_X(U)$, $t \in \mathcal{O}_X(V)$ (with $U, V$ nonempty) have the same germ at $\xi$ then $s|_{U \cap V} = t|_{U \cap V}$.
    
    If $f \in K$ we call $f$ a \hl{rational function} on $X$. The \hl{domain of definition} of $f$ is the union of all open sets $U$ occurring in the first position of tuples in the equivalence class $f$. This is a nonempty open set. If the domain of definition of $f$ is $V$, there is a unique section of $\mathcal{O}_X(V)$ whose germ at $\xi$ belongs to $f$. We also denote this section by $f$.
\end{remark}

Now we are ready to define Weil divisors:

\begin{definition}
\label{definition_1_1_2}
    Let $X$ be a Noetherian integral seperated scheme which is regular in codimension one. A \textit{\textcolor{ProcessBlue}{prime divisor}} on $X$ is a closed integral subscheme $Y$ of $X$ of dimension one. \par
    Then, a \textit{\textcolor{ProcessBlue}{Weil divisor}} is an element of the free abelian group $\Div{X}$, generated by all the prime divisors on $X$. So, an element $D \in \Div{X}$ is written in the form
    $$D = \sum n_iY_i$$ where $Y_i$ are the prime divisors, and $n_i$ are integers, only finitely many of them being non-zero. If $n_i \geq 0$ for all $i$, then $D$ is called an \textit{\textcolor{ProcessBlue}{effective divisor}}.
\end{definition}

\begin{remark}
    \label{remark_1_1_4}
    Let $Y$ be a prime divisor on $X$. Let $\eta \in Y$ be its generic point.  Then the local ring $\cc{O}_{X, \eta}$ is a DVR with quotient field $K$, the function field of $X$. The corresponding discrete valuation $v_Y$ is called the valuation of $Y$. (\cite{one}, II, Ex 4.5(a)). Now let $f \in K^*$ be a non-zero rational function on $X$. Then,. $v_Y(f) \in \bb{Z}$. If $v_Y(f) > 0$, then we say $f$ has a \textit{\textcolor{ProcessBlue}{zero along $Y$ of order $v_Y(f)$}}, and if $v_Y(f) < 0$, we say $f$ has a \textit{\textcolor{ProcessBlue}{pole along $Y$ of order $-v_Y(f)$}}.
\end{remark}

\begin{remark}
    \label{remark_1_1_5}
     If $Y$ is a prime divisor on $X$ with generic point $\eta$ and $f \in K^*$ then $v_Y(f) \geq 0$ if and only if $\eta$ belongs to the domain of definition of $f$, and $v_Y(f) > 0$ if and only if $f(\eta) = 0$ (that is, the germ of f with respect to  $eta$ resides in $\mathfrak{m}_\eta$).
\end{remark}

\begin{remark}
    \label{remark_1_1_6}
    If $X$ is a Noetherian integral seperated scheme which is regular in codimension one then so does any open subset $U \subseteq X$. If $Y$ is a prime divisor of $X$ then $Y \cap U$ is a prime divisor of $U$, provided it is nonempty. Moreover the assignment $Y \mapsto Y \cap U$ is injective since $Y = \overline{Y \cap U}$ (provided of course $Y \cap U \neq \emptyset$). If $Y$ is a prime divisor of $U$ then $Y$ is a prime divisor of $X$ and $\overline{Y} \cap U = Y$, so in fact there is a bijection between prime divisors of $U$ and prime divisors of $X$ meeting $U$.
\end{remark}
    
\begin{remark}
    \label{remark_1_1_7}
    If $A$ is a normal Noetherian domain then $X = \operatorname{Spec}A$ has the property $(*)$, since any affine scheme is separated. A prime divisor $Y \subset X$ is $V(\f{p})$ for a prime ideal $\f{p}$ of height 1. If $Q$ is the quotient field of $A$ there is a commutative diagram
    
    
    \begin{center}
        \begin{tikzpicture}
        \matrix(m)[matrix of math nodes,row sep=3.0em, column sep=5.5em,text height=1.5ex, text depth=0.25ex] 
        {  \cc{O}_{X, \f{p}} &  \cc{O}_{X, 0}\\
            A_{\f{p}} &  Q \\};
            \path[->](m-1-1) edge node[above]{} (m-1-2);
            \path[->](m-1-1) edge node[left]{} (m-2-1);
            \path[->](m-2-1) edge node[above]{} (m-2-2);
            \path[->](m-1-2) edge node[right]{} (m-2-2);
 \end{tikzpicture}
    \end{center}
    
    The discrete valuation $v_Y$ on $Q$ with valuation ring $A_p$ is defined on elements of $A_{\f{p}}$ by $v_Y(a/s) = \text{largest } k \geq 0 \text{ such that } a/s \in \f{p}^k A_{\f{p}}$.
\end{remark}
    
\begin{remark}
    \label{remark_1_1_8}
    If $Q$ is a field then the scheme $X = \operatorname{Spec}Q$ has the property $(*)$ but there are no prime divisors on $X$, so the case $Div X = 0$ can occur. But this is the \textit{only} way it can occur. If $X = \operatorname{Spec}A$ is an affine scheme satisfying $(*)$ with no prime divisors, then it follows from (\cite{one}, I, 1.11A) that $A$ must be a field.
\end{remark}

\begin{remark}
    \label{remark_1_1_9}
    If $X$ is a scheme which satisfies $(*)$ and has no prime divisors, then any point $x \in X$ must have an open neighborhood $V \cong \operatorname{Spec}A$ where $A$ is a field. Since every open subset of $X$ has no contained generic point, this is only possible if $X$ is a closed point (therefore isomorphic to the spectrum of its function field).
\end{remark}

\begin{lemma}
    \label{lemma_1_1_1}
    \textit{If $X$ satisfies $(*)$ then so does the scheme $\Spec(\mathcal{O}_{X,x})$ for any $x \in X$. If $f : \Spec(\mathcal{O}_{X,x}) \to X$ is canonical then $Y \mapsto f^{-1} Y$ gives an injective map from the set of prime divisors of $X$ containing $x$ to the set of prime divisors of $\Spec(\mathcal{O}_{X,x})$.}
\end{lemma}

\begin{proof}
    Suppose $X$ is a scheme satisfying $(*)$. Then $\mathcal{O}_{X,x}$ is a noetherian integral domain, so $\operatorname{Spec}(\mathcal{O}_{X,x})$ is clearly noetherian, integral and separated. Let $V \cong \operatorname{Spec} B$ be an open affine neighborhood of $x$ and suppose $\mathfrak{p}$ is the prime of $B$ corresponding to $x$. Then $\mathcal{O}_{X,x} \cong B_{\mathfrak{p}}$ and any local ring of the scheme $\operatorname{Spec}(\mathcal{O}_{X,x})$ is isomorphic to $(B_{\mathfrak{p}})_{\mathfrak{q}_{\mathfrak{p}}} \cong B_{\mathfrak{q}}$ for some prime $\mathfrak{q} \subset \mathfrak{p}$. It follows that $\operatorname{Spec}(\mathcal{O}_{X,x})$ is regular in codimension one.
\end{proof}

\begin{lemma}
    \label{lemma_1_1_2}
    \textit{Let $X$ satisfy $(*)$ and let $x \in X$. Then $x$ is contained in no prime divisor of $X$ if and only if $x$ is the unique generic point of $X$.}
\end{lemma}

\begin{proof}
    It is clear that the generic point is contained in no prime divisor of $X$. For the converse, let $x \in X$ be a point not contained in any prime divisor of $X$, and let $U \cong \operatorname{Spec} A$ be an affine open neighborhood of $x$, with prime $\mathfrak{p}$ corresponding to $x$. By assumption $\mathfrak{p}$ does not contain any prime ideal of height 1. Suppose $\mathfrak{p} \neq 0$, and let $0 \neq a \in \mathfrak{p}$. Then $\mathfrak{p}$ can be shrunk to a prime ideal minimal containing $(a)$, which must have height 1 by (\cite{one}, I, 1.11A), which is a contradiction. Hence $\mathfrak{p} = 0$ and $x$ is the generic point of $X$. 
\end{proof}

\begin{lemma}
     \label{lemma_1_1_3} Let $X$ be a Noetherian integral seperated scheme which is regular in codimension one, and let $f \in K^*$ be a non-zero rational function on $X$. Then $v_Y(f)$ vanishes for all but finitely many prime divisors $Y$ of $X$.
\end{lemma} 

\begin{proof}
    Let $U$ be the domain of definition of $f$. Then $Z = X \setminus U$ is a proper closed subset of $X$. Since $X$ is noetherian we can write $Z = Z_1 \cup \cdots \cup Z_n$ for closed irreducible $Z_i$. Thus $Z$ can contain at most a finite number of prime divisors of $X$, since any closed irreducible subset of $Z$ of codimension one in $X$ must be one of the $Z_i$. So it suffices to show that there are only finitely many prime divisors $Y \subseteq X$ meeting $U$ with $v_Y(f) \neq 0$. If $Y \cap U \neq \emptyset$ then $U$ must contain the generic point $\eta \neq Y$. Hence $v_Y(f) \geq 0$. Now $v_Y(f) > 0$ iff $\eta \in U \setminus X_f$, which is a proper closed subset of $U$ ($f \neq 0$ and therefore cannot be nilpotent by (\cite{one}, II, Ex 2.16) since $X$ noetherian and hence $U$ is quasi-compact$).$ Hence $v_Y(f) > 0$ iff $Y \cap U \subseteq U \setminus X_f$. But $U \setminus X_f$ is a proper closed subset of the noetherian space $U$, hence contains only finitely many closed irreducible subsets of codimension one of $U$. This proves the statement. 
\end{proof} 

\begin{definition}
    \label{definition_1_1_3}
    Let $X$ be a Noetherian integral seperated scheme which is regular in codimension one, and let $f \in K^*$. We define the \textit{\textcolor{ProcessBlue}{divisor of $f$}}, denoted by $(f)$,
    and $$(f) = \sum v_Y(f) \cdot Y$$ where the sum is taken over all prime divisors $Y$ of $X$. 
\end{definition}

\begin{remark}
    \label{remark_1_1_10}
    By \zcref{lemma_1_1_3}, this formal sum is finite and hence a Weil divisor by definition. Any divisor in $\Div{X}$ which is equal to a divisor of a function, is called a \textit{\textcolor{ProcessBlue}{principal divisor}}. This is a subgroup of $\Div{X}$.
\end{remark}

\begin{definition}
\label{definition_1_1_4}
    Let $X$ be a Noetherian integral seperated scheme which is regular in codimension one. Two divisors $D$ and $D'$ are said to be \hl{linearly equivalent} if $D - D'$ is a principal divisor. Then, the group we get by quotienting $\Div{X}$ by the group of principal divisors is called the \hl{divisor class group of $X$}, denoted by $\Cl{X}$. 
\end{definition}

This group is an invariant for every such scheme $X$, but it is in general difficult to calculate for arbitrary such scheme. We'd explore some special cases for which we know the divisor class group.

A ring $A$ is \hl{normal} if $A_{\f{p}}$ is a normal domain for every prime ideal $\f{p}$. We say that a scheme $X$ is \hl{normal} if all its local rings are normal domains, so the scheme $\Spec{A}$ is normal iff $A$ is a normal ring.

\begin{proposition}
    \label{proposition_1_1_1}
    Let $A$ be an integrally closed Noetherian integral domain. Then $$A = \bigcap_{\text{ht} \f{p} = 1} A_{\f{p}}$$
\end{proposition}

\begin{proof}
    See (\cite{two}, Theorem 38, p. 124).
\end{proof}

\begin{proposition}
    \label{proposition_1_1_2}
    Let $A$ be a Noetherian integral domain. Then $A$ is an UFD iff $X = \Spec{A}$ and $\Cl{X} = 0$.
\end{proposition}

\begin{proof}
    An UFD is integrally closed, so $X$ will be normal. On the other hand, $A$ is a UFD if and only if every prime ideal of height $1$ is principal. So what we must show is that if $A$ is an integrally closed domain, then every prime ideal of height $1$ is principal if and only if $\mathrm{Cl}(\mathrm{Spec}\, A) = 0$.

One way is easy: if every prime ideal of height $1$ is principal, consider a prime divisor $Y \subseteq X = \mathrm{Spec}\, A$. $Y$ corresponds to a prime ideal $\mathfrak{p}$ of height $1$. If $\mathfrak{p}$ is generated by an element $f \in A$, then clearly the divisor of $f$ is $1 \cdot Y$. Thus every prime divisor is principal, so $\mathrm{Cl}\, X = 0$.

For the converse, suppose $\mathrm{Cl}\, X = 0$. Let $\mathfrak{p}$ be a prime ideal of height $1$, and let $Y$ be the corresponding prime divisor. Then there is an $f \in K$, the quotient field of $A$, with $(f) = Y$. We will show that in fact $f \in A$ and $f$ generates $\mathfrak{p}$. Since $v_Y(f) = 1$, we have $f \in A_{Y}$, and $f$ generates $\mathfrak{p}_{Y}$. If $\mathfrak{p}' \subset A$ is any other prime ideal of height $1$, then $\mathfrak{p}'$ corresponds to a prime divisor $Y'$ of $X$, and $v_{Y'}(f) = 0$, so $f \in A_{Y'}$. Now the algebraic result in \zcref{proposition_1_1_1} below implies that $f \in A$. In fact, $f \in A \cap \mathfrak{p}_{Y} = \mathfrak{p}$. Now to show that $f$ generates $\mathfrak{p}$, let $g$ be any other element of $\mathfrak{p}$. Then $v_{Y}(g) \geq 1$ and $v_{Y}(g) > 0$ for all $Y' \neq Y$. Hence $v_{Y}(g/f) \geq 0$ for all prime divisors $Y'$ (including $Y$). Thus $g/f \in A_{Y'}$ for all $\mathfrak{p}'$ of height $1$, so by \zcref{proposition_1_1_1} again, $g/f \in A$. In other words, $g \in Af$, which shows that $\mathfrak{p}$ is a principal ideal, generated by $f$.
\end{proof}

\begin{cor}
    Let $A$ be a normal noetherian domain and set $X = \Spec{A}$. If $Q$ is the quotient
field of $A$ and $f \in Q^*$ then $(f)$ is an effective divisor if and only if $f \in A$.
\end{cor}

\begin{example}
    \label{example_1_1_2}
    If $X$ is affine $n$-space $\bb{A}^n_k$ over a field $k$, then $\mathrm{Cl}\, X = 0$. Indeed, $X = \mathrm{Spec}\, (k[x_1, \ldots, x_n])$, and the polynomial ring is a UFD.
\end{example}



Now, let $k$ be a field and $S = k[x_0, \ldots, x_n]$ ($n \ge 1$). Then $\mathbb{P}^n_k$ is an integral scheme of finite type over $k$, so the conclusions of (\cite{one}, II, Ex 3.20) apply. So for any irreducible closed subset $Y \subseteq \mathbb{P}^n_k$ we have
\[
\dim Y + \operatorname{codim}(Y, X) = \dim (\mathbb{P}^n_k) = n
\]

As we have studied before in projective spaces, we know that $\dim Y = \dim (S/I(Y)) - 1$. It follows that $\operatorname{ht}I(Y) = \operatorname{codim}(Y, X)$ and $\operatorname{codim} I(Y) = \dim Y + 1$. We know that $\mathbb{P}^n_k$ is a noetherian integral separated scheme. Regularity in codimension one follows from the fact that $\mathbb{P}^n_k$ is covered by affine opens $\operatorname{Spec} k[y_1, \ldots, y_n]$. Hence $\mathbb{P}^n_k$ satisfies the condition $(*)$ and we can talk about Weil divisors on $\mathbb{P}^n_k$.

The bijection between closed irreducible subsets of $\mathbb{P}^n_k$ and homogenous primes of $S$ other than $S_+$ identifies the prime divisors of $\mathbb{P}^n_k$ with the homogenous primes of height 1 (since $\operatorname{ht} S_+ = \operatorname{ht} (x_0, \ldots, x_n) = n + 1 > 1$). By (\cite{one}, I, 1.12A) every prime ideal of height 1 in $S$ is principal, and if $\mathfrak{p}$ is generated by an element $f$, then $f$ is necessarily an irreducible homogenous polynomial. Conversely every irreducible homogenous polynomial generates a homogenous prime of height 1 by (\cite{one}, I, 1.11A). So the map $f \mapsto V(f)$ sets up a bijection between associate classes of irreducible homogenous polynomials $f \in k[x_0, \ldots, x_n]$ with the prime divisors of $\mathbb{P}^n_k$. This means that the following definition makes sense:

\begin{definition}
    \label{definition_1_1_5}
    For a field $k$ and $n\geq 1$ let $Y \subseteq\mathbb{P}^n_k$ be a prime divisor. The \hl{degree of $Y$}, denoted $\deg{Y}$, is the degree of the associated irreducible homogenous polynomial (so $\deg{Y} \geq 1$). For any divisor $\displaystyle D = \sum_{i \in I}n_iY_i$ the degree of $D$ is $\displaystyle\deg{D} = \sum_{i \in I} n_i \cdot \deg{Y_i}$. Similarly we define the degree of a prime divisor $Y \subset \mathbb{A}^n_k$ to be the degree of the associated irreducible polynomial, and the degree of a divisor on $\mathbb{A}^n_k$ as above. It is clear that for divisors $D, D'$ on affine or projective space
$$\deg(D + D') = \deg{D} + \deg{D'}$$.
\end{definition}
Now, let $X = \mathbb{P}^n_k$ be projective $n$-space over a field $k$ for $n \geq 1$. Given nonzero homogeneous $g, h \in S$ of the same degree the corresponding principal divisor is
\[
(g/h) = \sum_{\substack{\mathfrak{p} \text{ homogeneous} \\ \operatorname{ht}\mathfrak{p}=1}} v(\mathfrak{p})(g/h) \cdot V(\mathfrak{p}) = \sum_{f} \left( v_f(g) - v_f(h) \right) \cdot V(f)
\]
where the second sum is over the set of equivalence classes of irreducible homogeneous polynomials under the associate relation, and we pick a single $f$ from each class.

\begin{proposition}
    \label{proposition_1_1_3} \textit{Let $X$ be the projective space $\mathbb{P}^n_k$ over a field $k$. Let $H$ be the hyperplane $x_0 = 0$. Then:}
\begin{enumerate}
    \item[(a)] \textit{if $D$ is any divisor of degree $d$, then $D \sim dH$;}
    \item[(b)] \textit{for any $f \in K^*$, $\deg(f) = 0$;}
    \item[(c)] \textit{the degree function gives an isomorphism $\deg: \mathrm{Cl}\, X \to \mathbb{Z}$.}
\end{enumerate}
\end{proposition}

\begin{proof} Let $S = k[x_0, \ldots, x_n]$ and $K$ the function field $\mathcal{O}_{X,0}$ of $X$. If $f \in K^*$ then $f$ corresponds to a quotient $g/h$ of two nonzero homogeneous polynomials $g, h \in S$ of the same degree. If we factor $g, h$ as $g = u p_1^{n_1} \cdots p_r^{n_r}$ and $h = v p_1^{m_1} \cdots p_r^{m_r}$ for $u, v \in k$ and irreducible polynomials $p_i$, then the $p_i$ must be homogeneous (we allow some zero indices to get the occurring $p_i$ the same in both cases) and the principal divisor $(f)$ is defined by

\[
(f) = \sum_{i=1}^{r} (n_i - m_i) \cdot Y_i, \qquad Y_i = V(p_i)
\]

Hence
\[
\deg(f) = \sum_{i=1}^{r} (n_i - m_i)\deg(p_i) = \sum_{i=1}^{r} n_i\deg(p_i) - \sum_{i=1}^{r} m_i \deg(p_i) = \deg(g) - \deg(h) = 0
\]

Which proves (b). To prove (a), let $D = \sum_{i=1}^r n_i \cdot D(p_i)$ be any nonzero effective divisor of degree $d$ with $n_i > 0$ and the $p_i$ homogeneous irreducibles. Then $p_1^{n_1} \cdots p_r^{n_r}/x_0^d \in S(0)$ and the corresponding principal divisor is $D - dH$, where $H = V(x_0)$, which shows that $D \sim dH$ (of course there is nothing special about $x_0$, we could have used any $x_i$). It follows immediately that any divisor of degree zero is principal.

Taking degrees defines a morphism of abelian groups $Div X \longrightarrow \mathbb{Z}$, and we have just shown the kernel of this map consists of the principal divisors. Since $\deg(dH) = d$ for any $d \in \mathbb{Z}$ we obtain the required isomorphism $Cl X \longrightarrow \mathbb{Z}$. 
\end{proof}

\begin{proposition}
    \label{proposition_1_1_4}
    \textit{Let $X$ be a Noetherian integral seperated scheme which is regular in codimension one, let $Z$ be a proper closed subset of $X$, and let $U = X - Z$. Then:}
\begin{enumerate}
  \item[(a)] \textit{there is a surjective homomorphism $\mathrm{Cl}\, X \to \mathrm{Cl}\, U$ defined by $D = \sum n_i Y_i \mapsto \sum n_i (Y_i \cap U)$, where we ignore those $Y_i$ on which $Y_i \cap U$ which are empty;}
  \item[(b)] \textit{if $\operatorname{codim}_X (Z) \ge 2$, then $\mathrm{Cl}\, X \to \mathrm{Cl}\, U$ is an isomorphism;}
  \item[(c)] \textit{if $Z$ is an irreducible subset of codimension $1$, then there is an exact sequence
\[
\mathbb{Z} \to \mathrm{Cl}\, X \to \mathrm{Cl}\, U \to 0,
\]
where the first map is defined by $1 \mapsto Z$.}
\end{enumerate}
\end{proposition} 

\begin{proof}
   (a) We have already noted that if $X$ satisfies $(*)$ so does $U$, so the group $\Cl\,U$ is defined. We have also noted that $Y \mapsto Y \cap U$ gives a bijection between the prime divisors of $X$ meeting $U$ and the prime divisors of $U$. Since $Z$ is a proper closed subset of the noetherian space $X$, we can write it as a union $Z_1 \cup \cdots \cup Z_n$ of irreducible components, so there is only a finite number of prime divisors of $X$ not meeting $U$.

Let $\varphi : \mathrm{Div}\,{X} \to \mathrm{Div}\,{U}$ be the morphism of abelian groups induced by sending those $Y$ with $Y \cap U \neq \emptyset$ to $Y \cap U$ and all other $Y$ to zero (if $\mathrm{Div}\,X = 0$ we just take the zero map). We claim that $\varphi$ sends principal divisors to principal divisors. It clearly sends \textit{any} divisor whose support does not include prime divisors meeting $U$ to zero, so suppose $\xi$ is the generic point of $X$, $K = \mathcal{O}_{X,\xi}$ and $f \in K^*$ with $v_Y(f) \neq 0$ for some prime divisor $Y$ with $Y \cap U \neq \emptyset$. For such $Y$ with generic point $\eta$ there is a commutative diagram

\begin{center}
        \begin{tikzpicture}
        \matrix(m)[matrix of math nodes,row sep=3.0em, column sep=5.5em,text height=1.5ex, text depth=0.25ex] 
        {  Q &  K\\
            \cc{O}_{U, \eta} &  \cc{O}_{X, \eta} \\};
            \path[->](m-1-1) edge node[above]{$\simeq$} (m-1-2);
            \path[->](m-2-1) edge node[left]{} (m-1-1);
            \path[->](m-2-1) edge node[above]{$\simeq$} (m-2-2);
            \path[->](m-2-2) edge node[right]{} (m-1-2);
 \end{tikzpicture}
\end{center}

If $g$ is the image of $f$ in $Q^*$ then $v_{Y \cap U}(g) = v_Y(f)$ so it is clear that the image under $\varphi$ of the principal divisor $(f)$ is the principal divisor $(g)$, as required (in fact $\varphi$ maps the principal divisors of $X$ \textit{onto} the principal divisors of $U$). So we induce the desired surjective morphism of abelian groups $\Cl\,X \to \Cl\,U$.

(b) Let $Z = Z_1 \cup \cdots \cup Z_n$ be the irreducible components of $Z$. If a prime divisor of $X$ does not meet $U$ then it must be one of the $Z_i$. Since $\mathrm{codim}(Z, X) = \displaystyle\min_{i \in I}\{\mathrm{codim}(Z_i, X)\}$ by assuming $\mathrm{codim}(Z, X) \geq 2$ we are excluding this possibility. Thus every prime divisor of $X$ must meet $U$, so there is a bijection between the prime divisors of $U$ and $X$ and so it is easy to see that the kernel of $\Cl\,X \to \Cl\,U$ is an isomorphism.

(c) If there are no prime divisors in $U$ then $Z$ is the only prime divisor of $X$, so 
$\mathrm{Im}(\mathbb{Z} \to \Cl\,X) = \Cl\,X = \ker(\Cl\,X \to \Cl\,U)$ so the sequence is exact. Otherwise it is easy to see that the kernel of $\Cl\,X \to \Cl\,U$ consists of divisors whose support is contained in $\{Z\}$, so the sequence is exact.
\end{proof}
\begin{example}
    \label{example_1_1_3}
    For a field $k$ the prime divisors of $\mathbb{P}^2_k$ are the irreducible curves. Given an irreducible curve $Y$ of degree $d$, $\Cl(\mathbb{P}^2_k \setminus Y) \cong \bb{Z} / {d\bb{Z}}$. This follows immediately from \zcref{proposition_1_1_3} and \zcref{proposition_1_1_4}. If $d = 1$ then $Y$ is a line and $\Cl(\mathbb{P}^2_k \setminus Y) = 0$. Otherwise $Y \neq V(x_0)$ and we can describe the surjective morphism $\bb{Z} \to \Cl\,U$ with kernel $(d)$ by $n \mapsto n \cdot (V(x_0) \cap U)$ where $U = \mathbb{P}^2_k \setminus Y$.
\end{example}

\begin{example}
    \label{example_1_1_4}
Let $k$ be a field, let $A = k[x, y, z] / (xy - z^2)$, and let $X = \operatorname{Spec} A$. Then $X$ is an affine quadric cone in $\mathbb{A}^3$. We will show that $\operatorname{Cl} X = \mathbb{Z} / 2\mathbb{Z}$, and that it is generated by a ruling of the cone, say $Y: y = z = 0$.

First note that $Y$ is a prime divisor, so by \zcref{proposition_1_1_4} we have an exact sequence
\[
\mathbb{Z} \longrightarrow \operatorname{Cl} X \longrightarrow \operatorname{Cl}(X - Y) \longrightarrow 0,
\]
where the first map sends $1 \mapsto 1 \cdot Y$. Now $Y$ can be cut out set-theoretically by the function $y$. In fact, the divisor of $y$ is $2 \cdot Y$, because $y=0 \implies z^2 = 0$, and $z$ generates the maximal ideal of the local ring at the generic point of $Y$. Hence $X - Y = \operatorname{Spec} A_y$. Now $A_y = k[x, y, z, y^{-1}] / (xy - z^2)$; in this ring $x = y^{-1} z^2$, so we can eliminate $x$, and find $A_y \cong k[y, y^{-1}, z]$. This is a UFD, so by \zcref{proposition_1_1_2}, $\operatorname{Cl}(X - Y) = 0$.

Thus we see that $\operatorname{Cl} X$ is generated by $Y$, and that $2 \cdot Y = 0$. It remains to show that $Y$ itself is not a principal divisor. Since $A$ is integrally closed (\cite{one}, II, Ex. 6.4), it is equivalent to show that the prime ideal of $Y$, namely $\mathfrak{p} = (y, z)$, is not principal (Check the proof of \zcref{proposition_1_1_2}). Let $\mathfrak{m} = (x, y, z)$, and note that $\mathfrak{m} / \mathfrak{m}^2$ is a 3-dimensional vector space over $k$ generated by $\overline{x}, \overline{y}, \overline{z}$, the images of $x, y, z$. Now $\mathfrak{p} \subseteq \mathfrak{m}$, and the image of $\mathfrak{p}$ in $\mathfrak{m} / \mathfrak{m}^2$ contains $\overline{y}$ and $\overline{z}$. Hence $\mathfrak{p}$ cannot be a principal ideal.

\end{example}

\section{Divisors on Curves}

We will illustrate the notion of the divisor class group further by paying special attention to the case of divisors on curves. We will define the degree of a divisor on a curve, and we will show that on a complete nonsingular curve, the degree is stable under linear equivalence. In this section $k$ denotes an algebraically closed field.

To begin with, we need some preliminary information about curves and morphisms of curves. So we recall our conventions about terminology of curves:

\begin{definition}
    \label{definition_1_2_1}
    A scheme $X$ is \hl{nonsingular} if all the local rings of $X$ are regular local rings. A \hl{variety over $k$} is an integral separated scheme $X$ of finite type over $k$. A \hl{curve} is a variety of dimension one. If $X$ is proper over $k$, we say that $X$ is \hl{complete}. If $Y$ is a nonsingular curve in the sense of (\cite{one}, I), then $t(Y)$ is a nonsingular curve in the present sense.
\end{definition} 

\begin{lemma}
    \label{lemma_1_2_1}
    Let $X$ be a variety over $k$. \textit{Then a point $x \in X$ is closed if and only if its residue field is $k$, and any morphism $f : X \longrightarrow Y$ of varieties over $k$ maps closed points to closed points.}
\end{lemma}

\begin{proof}
    Omitted.
\end{proof}

\begin{lemma}
    \label{curve_prop}
     Let $X$ be a curve over $k$. Then
\begin{itemize}
    \item[(i)] $x \in X$ is closed if and only if $\dim \mathcal{O}_{X,x} = 1$ and the only nonclosed point is the generic point.
    \item[(ii)] $X$ has the finite complement topology.
    \item[(iii)] $K(X)$ is a finitely generated field extension of $k$ of transcendence degree $1$.
\end{itemize}
\end{lemma}

\begin{proof}
    (i) If $x$ is closed then by (\cite{one}, II, Ex 3.20), $\dim \mathcal{O}_{X, x} = \dim X = 1$. Conversely if $\dim \mathcal{O}_{X, x} = 1$, let $Y = \overline{\{x\}}$. This is an irreducible closed subset of $X$, so again by (\cite{one}, II, Ex 3.20) we have $\dim Y + \mathrm{codim}(Y, X) = 1$. But $\mathrm{codim}(Y, X) = \dim \mathcal{O}_{X, x}$, by (\cite{one}, II, Ex 3.6) so $\dim Y = 0$. Since an irreducible closed subset of a scheme has a unique generic point, it follows that $Y = \{ x \}$, as required. If $y \in X$ is any point, it follows from (\cite{one}, II, Ex 3.20) that $\dim \mathcal{O}_{X, y} = \dim X - \dim \{ y \} \leq \dim X$ so if $y$ is not closed, $\dim \mathcal{O}_{X, y} = 0$. But then $\dim \{ y \} = \dim X$ and by Ex I, 1.10 this is only possible if $\{y \} = X$. Hence $y = \xi$ is the only nonclosed point. \par
    (ii) $X$ is an irreducible space of dimension 1, and if $Y$ is a proper closed irreducible subset then $\dim Y = 0$. Since $\xi \neq y$ it follows from (i) that $Y$ must be a point. Since $X$ is noetherian any closed subset is either $X$ or a finite union of points. \par
    (iii) Follows immediately from (\cite{one}, II, Ex 3.20).
\end{proof}

In particular, if $Y$ is a curve in the sense of (\cite{one}, I), then when we form the scheme $t(Y)$ we only add one point: the generic point, which corresponds to the irreducible subset of $Y$ consisting of the whole space. This is obvious anyway, since any curve has the finite complement topology, so the only closed irreducible subsets are points and the whole space.

Let $X$ be a nonsingular curve over $k$. Then if $x \in X$ is a closed point, $\mathcal{O}_{X, x}$ is a discrete valuation ring with quotient field $K(X)$, so there is a discrete valuation $v_x$ on $K(X)$ with valuation ring $\mathcal{O}_{X,x}$. If $f \in K(X)^*$ then we can represent $f$ by a section on the domain of definition $U_f$ of $f$, and it is not difficult to see that $v_x(f) \geq 0$ if and only if $x \in U_f$.

\begin{proposition}
    \textit{Let $X$ be a nonsingular curve over $k$ with function field $K$. Then the following conditions are equivalent:}
\begin{itemize}
    \item[(a)] $X$ is projective;
    \item[(b)] $X$ is complete;
    \item[(c)] $X \cong t(C_K)$ as schemes over $k$, where $C_K$ is the abstract nonsingular curve of (\cite{one}, I, Section 6) and $t$ is the functor from varieties to schemes of (II, 2.6).
\end{itemize}
\end{proposition}

\begin{proof}
    (a) $\Rightarrow$ (b) Follows from (\cite{one}, II, 4.9). \par
    (b) $\Rightarrow$ (c) If $x \in X$ is a closed point then $\dim \mathcal{O}_{X, x} = \dim X = 1$ by (\cite{one}, II, Ex 3.20), so $\mathcal{O}_{X, x}$ is a discrete valuation ring. Considered as a subring of $K$, $\mathcal{O}_{X, x}$ determines a discrete valuation $v_x$ of $K/k$. If $v$ is a discrete valuation of $K/k$, then by (\cite{one}, II, Ex 1.5), $v$ has a unique center $x \in X$. That is, the valuation ring $R$ of $v$ dominates $\mathcal{O}_{X,x}$. Since $R \subset K$, it follows from Lemma 1.2.2 that $x$ must be a closed point. Therefore $\mathcal{O}_{X, x}$ is a discrete valuation ring, which is maximal under domination, so $R = \mathcal{O}_{X, x}$. Since the center of $v$ is unique, this gives a bijection between the closed points of $X$ and the discrete valuations of $K/k$, which are the points of $C_K$.

    The points of $t(C_K)$ and the points of $X$ thus each have a generic point, so by matching generic points we get a bijection $X \leftrightarrow t(C_K)$. Since both have the finite complement topology, this is trivially a homomorphism. By (\cite{one}, II, Ex 3.6) there is an injective ring morphism $\mathcal{O}_X(U) \longrightarrow K$ for any nonempty open $U \subset X$. Clearly $\mathcal{O}_X(U) \subset \displaystyle\bigcap_{x \in U} \mathcal{O}_{X,x}$, and it is not hard to see this is an equality. So $X \cong t(C_K)$ is actually an isomorphism of schemes. \par
    (c) $\Rightarrow$ (a) Immediately follows from (\cite{one}, I, 6.9).
\end{proof}

This result actually shows that for a given function field, there is only one projective nonsingular curve upto isomorphism.

\begin{lemma}
    \label{lemma_1_2_3}
    If $f : X \to Y$ is a finite morphism of curves over $k$, then $f$ maps the generic point
of $X$ to the generic point of $Y$. The degree of the field extension $K(X) / K(Y)$ is finite.
\end{lemma}

\begin{proof}
    Let $\xi \in X, \eta \in Y$ be the generic points and suppose that $f(\xi) = z \neq \eta$. Then $z$ is a closed point, so $f^{-1}(z)$ is closed, and therefore $f^{-1}(z) = X$. But $f$ is finite, and a finite morphism is quasi-finite by (\cite{one}, II, Ex. 3.5), so this is a contradiction. Using the canonical injective morphism of $k$-algebras $K(Y) \longrightarrow K(X)$ we consider $K(Y)$ as a subfield of $K(X)$. Since $K(X), K(Y)$ both have transcendence degree 1 over $k$, it follows that $K(X)$ is algebraic over $K(Y)$. Since $K(X)$ is finitely generated over $k$, it is finitely generated over $K(Y)$, so $K(X)/K(Y)$ is a finite extension.
\end{proof}

\begin{definition}
    \label{definition_1_2_2}
    Let $f : X \longrightarrow Y$ be a finite morphism of curves over $k$. The \hl{degree of $f$} is the degree of the field extension $[K(X) : K(Y)]$.
\end{definition}

\begin{lemma}
    \label{lemma_1_2_4}
    \textit{Let $A$ be a normal domain, $K$ its quotient field, $F$ a finite algebraic extension of $K$, and $A'$ the integral closure of $A$ in $F$. Then $F$ is the quotient field of $A'$.}
\end{lemma}

\begin{proof}
    See (\cite{five}, Theorem 7, Chapter 5.)
\end{proof}

\begin{lemma}
    \label{lemma_1_2_5}
    \textit{Let $A$ be a Dedekind domain with quotient field $K$. Then the valuation rings of $K$ containing $A$ are precisely the subrings $A_{\mathfrak{p}}$ with $\mathfrak{p}$ a prime ideal of $A$.}
\end{lemma}

\begin{proof}
    It is not hard to see that the subrings $A_{\mathfrak{p}}$ of $A$ are all distinct, and since $A$ is Dedekind they are all valuation rings of $K$ (if $\mathfrak{p} \neq 0$ then $A_{\mathfrak{p}}$ is a discrete valuation ring of $K$). If $(V, m)$ is a valuation ring of $K$ containing $A$ then $m \cap A = \mathfrak{p}$ is a prime ideal of $A$. If $\mathfrak{p} = 0$ then $V = K$. Otherwise $V$ must dominate the valuation ring $A_{\mathfrak{p}}$, and hence $V = A_{\mathfrak{p}}$.
\end{proof}

\begin{example}
    \label{example_1_2_1}
    Take $A = k[x]$ for a field $k$. Then the nonzero primes of $A$ are all of the form $\mathfrak{p} = (f)$ for an irreducible polynomial $f$. The valuation on $K = k(x)$ corresponding to $A_{\mathfrak{p}}$ takes a quotient $g/h$ and spits out the power of $f$ dividing $g$ minus the power of $f$ dividing $h$. Lemma 1.2.5 shows that these are the only valuations of $K/A$. This example shows intuitively why $V$ is a valuation ring of $K$, since any quotient $g/h$ can be reduced until $f$ divides at most one of $g, h$. So it is clear that $g/h, h/g$ must both belong to $A$. \par

If $A$ is not Dedekind then the local rings may not be valuation rings of $K$. Take for example $A = k[x, y]/(xy - z)$ and $\mathfrak{p} = (z, y)$. Then $a = z/y$ is an element of the quotient field with $a \notin A_{\mathfrak{p}}, a^{-1} \notin A_{\mathfrak{p}}$.
\end{example}

\begin{proposition}
    \label{proposition_1_2_2}
    Let $X$ be a complete nonsingular curve over $k$, let $Y$ be any nonsingular curve over $k$ and let $f : X \longrightarrow Y$ be a morphism of schemes over $k$. Then either 
    \begin{enumerate}
        \item $f(X)$ = a point, or,
        \item $K(X)$ is a finite extension field of $K(Y)$, $Y$ is complete and $f$ is a finite morphism.
    \end{enumerate}
\end{proposition}

\begin{proof}
    It follows from (\cite{one}, II, Ex 4.4) that $f$ is proper, and hence $f(X)$ is closed. Since $X$ is irreducible, so is $f(X)$. So it is clear that one of (1), (2) must hold. Suppose that (2) holds and let $Z \longrightarrow Y$ be the image of $f$. The underlying topological space of $Z$ is $f(X)$, so $Z \longrightarrow Y$ is a surjective closed immersion. Since $Y$ is reduced, this means that $Z \longrightarrow Y$ is an isomorphism. By (\cite{one}, II, Ex 4.4) the composite $Z \longrightarrow Y \longrightarrow S$ is proper, so $Y$ is complete.

Since $Y$ is infinite and $f$ is surjective, $f$ must map the generic point of $X$ to the generic point of $Y$, which gives an injection of $k$-algebras $K(Y) \subseteq K(X)$. Using the argument of Lemma 1.2.3 we see that $K(X)/K(Y)$ is a finite field extension. For $x \in X$ taking the intersection $\mathcal{O}_{X,x} \cap K(Y)$ gives a valuation ring of $K(Y)$, which dominates $\mathcal{O}_{Y, f(x)}$ since the morphism $\mathcal{O}_{Y, f(x)} \longrightarrow \mathcal{O}_{X, x}$ is local. Therefore $\mathcal{O}_{X,x} \cap K(Y) = \mathcal{O}_{Y, f(x)}$ since valuation rings are maximal with respect to domination. Note that since $K(X)/K(Y)$ is algebraic, if $x$ is not the generic point $\mathcal{O}_{X,x} \subseteq K(X)$ and therefore $\mathcal{O}_{Y, f(x)} \subseteq K(Y)$. That is, the fiber of $f$ over the generic point of $Y$ consists precisely of the generic point of $X$.

Now suppose that $Y$ is nonsingular. Let $V \cong \operatorname{Spec} B$ be an affine open subset of $Y$, with $B$ a finitely generated $k$-domain. Using (\cite{one}, II, Ex 3.20e) and the fact that $Y$ is nonsingular, we see that $B$ is a Dedekind domain. We can identify $B$ with a subring of $K(Y)$ (which is then the quotient field of $B$) and hence with a subring of $K(X)$. Let $A$ be the integral closure of $B$ in $K(X)$. Then using (\cite{one}, I, 6.3A) and (\cite{one}, I, 3.9.4) we see that $A$ is a Dedekind finitely generated $k$-domain, which is also a finitely generated $B$-module. By Lemma 1.2.4 $K(X)$ is the quotient field of $A$.

Therefore we can find a nonsingular curve $Z \subseteq \mathbb{A}^n$ for some $n \geq 1$ (in the sense of (\cite{one}, I)) with coordinate ring $A(Z) \cong A$ as $k$-algebras, and function field $K(Z)$ $k$-isomorphic to $K(X)$. By (\cite{one}, I, 6.7) there is an isomorphism of ``abstract nonsingular curves'' $Z \cong U$, where $U$ is an open subset of the abstract nonsingular curve $C_{K(X)}$. Hence $\iota(Z) \cong \iota(U) \subseteq \iota(C_{K(X)}) \cong t(C_{K(X)})$ as schemes over $k$. Then there is an isomorphism of schemes $\iota(Z) \cong \operatorname{Spec} A$ over $k$, so finally $\operatorname{Spec} A$ is isomorphic to an open subset of $\iota(C_{K(X)}) \subseteq X$. This isomorphism sends a nonzero prime $\mathfrak{p} \in \operatorname{Spec} A$ to the discrete valuation ring $A_\mathfrak{p}$ of $K(X)$ and then to the corresponding point of $X$.

Since $X, Y$ are both complete, for both curves there is a bijection between discrete valuation rings over $k$ and closed points. The closed points $y \in V$ correspond to subrings $\mathcal{O}_{Y, y} \subseteq K(Y)$, which by Lemma 1.2.5 are precisely the discrete valuation rings of $K(Y)$ containing $B$. Therefore, if $x \in X$ we have $f(x) \in V$ if and only if $\mathcal{O}_{X,x} \supseteq B$, and since $A$ is the integral closure of $B$, this is if and only if $\mathcal{O}_{X,x} \supseteq A$. Since $A$ is the intersection of all the valuation rings of $K(X)$ containing $A$ (See \cite{four}, Corollary 5.22), we have
\[
A = \bigcap_{x \in f^{-1} V} \mathcal{O}_{X,x} = \mathcal{O}_X(f^{-1} V)
\]
We have constructed an isomorphism schemes over $k$ of $\Spec{A}$ with the open subset $f^{-1}(V)$ of $X$.
By construction $A$ is a finitely generated $B$-module, and we can cover $Y$ with affine open sets of the form $V$, so this shows that $f$ is finite.
\end{proof}

Now we are ready to move to the study of divisors on curves. If $X$ is a nonsingular curve, then $X$ satisfies the condition $(*)$ used above, so we can talk about Weil divisors on $X$. A prime divisor is just a closed point, so an arbitrary divisor can be written $D = \sum n_i P_i$ where the $P_i$ are closed points, and $n_i \in \mathbb{Z}$. We define the \hl{degree} of $D$ to be $\sum n_i$. Clearly this defines a morphism of abelian groups $\deg : \operatorname{Div} X \longrightarrow \mathbb{Z}$.

\begin{definition}
    \label{definition_1_2_3}
    If $f: X \longrightarrow Y$ is a finite morphism of nonsingular curves over $k$, we define a morphism of abelian groups $f^* : \operatorname{Div} Y \longrightarrow \operatorname{Div} X$ as follows. For any closed point $Q \in Y$, let $t \in \mathcal{O}_{Y, Q}$ be a \hl{local parameter} at $Q$, which is an element of $K(Y)$ with $v_Q(t) = 1$, where $v_Q$ is the valuation corresponding to the discrete valuation ring $\mathcal{O}_{Y,Q}$. We define
\[
f^* Q = \sum_{f(P) = Q} v_P(l) \cdot P
\]
Since $f$ is a finite morphism this is a finite sum (\cite{one}, II, Ex 3.5), so we get a divisor on $X$. Note that $f^* Q$ is independent of the choice of the local parameter $l$. If $l' = u l$ is another local parameter, then $l' = u l$ where $u$ is a unit in $\mathcal{O}_{Y,Q}$. For any point $P \in X$ with $f(P) = Q$, $u$ will map to a unit in $\mathcal{O}_{X, P}$, so $v_P(u l) = v_P(l)$. We extend the definition by linearity to all divisors on $Y$.
\end{definition}

\begin{remark}
    \label{remark_1_2_1}
    Let $P \in X, Q \in Y$ be closed points with $f(P) = Q$, and let $l$ be a local parameter at $Q$. If $0 \neq g \in \mathcal{O}_{Y, Q}$ then $v_Q(g)$ is the largest integer $k \geq 0$ for which $g \in m_Q^k$, so it is clear that $v_P(g) \geq v_Q(g)$. We claim that $v_P(g) = v_P(l) v_Q(g)$. This is trivial if $v_Q(g) = 0$ since then $v_P(g) = 0$. If $v_Q(g) = k \geq 1$, then $g = u l^k$ where $u$ is a unit in $\mathcal{O}_{Y, Q}$. Therefore $u$ is also a unit in $\mathcal{O}_{X,P}$, so $v_P(g) = v_P(u l^k) = v_P(l^k) = k v_P(l)$, as required. It follows that the image under $f^*$ of the principal divisor $(g)$ is the principal divisor $(g)$ obtained from $g \in K(X)$. Hence $f^*$ induces a morphism of abelian groups $f^* : \operatorname{Cl} Y \longrightarrow \operatorname{Cl} X$.
\end{remark}

\begin{proposition}
     Let $f : X \to Y$ be a finite morphism of nonsingular curves. Then for any divisor $D$ on $Y$ we have $\deg f^* D = \deg f \cdot \deg D$.
\end{proposition}

\begin{proof}
    It will be sufficient to show that for any closed point $Q \in Y$ we have $\deg f^* Q = \deg f$. Let $V = \mathrm{Spec} B$ be an open affine subset of $Y$ containing $Q$. Let $A$ be the integral closure of $B$ in $K(X)$. Then, as in the proof of Proposition 1.2.2, $U = \mathrm{Spec} A$ is the open subset $f^{-1} V$ of $X$. Let $\mathfrak{m}_Q$ be the maximal ideal of $Q$ in $B$. We localize both $B$ and $A$ with respect to the multiplicative system $S = B - \mathfrak{m}_Q$, and we obtain a ring extension $\mathcal{O}_{Y, Q} \subseteq A'$, where $A'$ is a finitely generated $\mathcal{O}_{Y, Q}$-module. Now $A'$ is torsion-free, and has rank equal to $r = [K(X) : K(Y)]$, so $A'$ is a free $\mathcal{O}_{Y,Q}$-module of rank $r = \deg f$. If $t$ is a local parameter at $Q$, it follows that $A' / t A'$ is a $k$-vector space of dimension $r$.

On the other hand, the points $P_i$ of $X$ such that $f(P_i) = Q$ are in 1-1 correspondence with the maximal ideals $\mathfrak{m}_i$ of $A'$, and for each $i$, $A'_i = \mathcal{O}_{X, P_i}$. Clearly $tA' = \bigcap_i t A'_i \cap A'$, so by the Chinese remainder theorem
\[
\dim_k A' / t A' = \sum_i \dim_k A'_i / (t A'_i \cap A').
\]
But
\[
A'_i / (t A'_i \cap A') = A'_i / t A'_i = \mathcal{O}_{X, P_i} / t \mathcal{O}_{X, P_i},
\]
so the dimensions in the sum above are just equal to $v_{P_i}(t)$. But $f^* Q = \sum v_{P_i}(t) \cdot P_i$, so we have shown that $\deg f^* Q = \deg f$ as required.
\end{proof}

\begin{cor}
    A principal divisor on a complete nonsingular curve $X$ has degree zero. Consequently, the degree function induces a surjective homomorphism $\deg \colon \mathrm{Cl} X \to \mathbb{Z}$.
\end{cor} 

\begin{proof}
    Let $f \in K(X)^*$. If $f \in k^*$ then $(f) = 0$, so there is nothing to prove. If $f \notin k$, then the inclusion of fields $k(f) \subseteq K(X)$ induces a finite morphism $\varphi : X \to \mathbb{P}^1$. It is a morphism by (\cite{one}, I, 6.12), and it is finite by Proposition 1.2.2. Now $(f) = \varphi^*(0) - \varphi^*(\infty)$. Since $\varphi^*(0) - \varphi^*(\infty)$ is a divisor of degree 0 on $\mathbb{P}^1$, we conclude that $(f)$ has degree 0 on $X$. Thus the degree of a divisor on $X$ depends only on its linear equivalence class, and we obtain a homomorphism $\mathrm{Cl} X \to \mathbb{Z}$ as stated. It is surjective, because the degree of a single point is 1.
\end{proof}

\section{Cartier Divisors}



Now we want to extend the notion of divisor to an arbitrary scheme. It turns out that using the irreducible subvarieties of codimension one doesn't work very well. So instead, we take as our point of departure the idea that a divisor should be something which locally looks like the divisor of a rational function. This is not exactly a generalisation of the Weil divisors, but it gives a good notion to use on arbitrary schemes.

\begin{definition}
    \label{definition_1_3_1}
    Let $X$ be a scheme. For each open set $U \subseteq X$ let $S(U)$ denote the set of elements of $\mathcal{O}_X(U)$ which are regular in $\mathcal{O}_{X, x}$ for every $x \in U$. For nonempty $U$ this is a multiplicatively closed subset, and we define $Q(U) = S(U)^{-1} \mathcal{O}_X(U)$. If $U \subseteq V$ then restriction maps $S(V)$ to $S(U)$, so there is a morphism of rings $Q(V) \rightarrow Q(U)$ defined by $a/s \mapsto a|_U / s|_U$. Thus defined $Q$ is a presheaf of commutative rings, whose associated sheaf of commutative rings $\mathcal{K}_X$ we call the sheaf of total quotient rings of $X$. On an arbitrary scheme, the sheaf $\mathcal{K}_X$ replaces the concept of function field of an integral scheme. We simply write $\mathcal{K}$ for $\mathcal{K}_X$ if there is no chance of confusion.
\end{definition}

\begin{lemma}
    \label{lemma_1_3_1}
    Let $(X, \mathcal{O}_X)$ be a ringed space and let $M(U)$ be the set of invertible elements of the ring $\mathcal{O}_X(U)$. Then $M$ is a sheaf of (multiplicative) abelian groups.
\end{lemma}

\begin{proof}
    Omitted.
\end{proof}

We denote by $\cc{O}^*$ the sheaf of (multiplicative abelian groups) of invertible elements in the sheaf of rings $\mathcal{O}_X$. Similarly $\mathcal{K}^*$ is the sheaf of invertible elements in $\mathcal{K}$. There are canonical morphisms of presheaves of rings $\mathcal{O}_X \rightarrow Q$ and $Q \rightarrow \mathcal{K}$. The composite $\mathcal{O}_X \rightarrow \mathcal{K}$ makes $\mathcal{K}$ into a sheaf of $\mathcal{O}_X$-algebras, and gives rise to a morphism of sheaves of abelian groups $\cc{O}^* \rightarrow \mathcal{K}^*$.

\begin{lemma}
    \label{lemma_1_3_2}
     The morphism $\mathcal{O}_X \rightarrow \mathcal{K}$ is a monomorphism of sheaves of commutative rings, and $\cc{O}^* \rightarrow \mathcal{K}^*$ is a monomorphism of sheaves of abelian groups.
\end{lemma}

\begin{proof}
    The second claim follows immediately from the first. For $x \in X$ we need to show that the composite $\mathcal{O}_{X, x} \rightarrow Q_x \rightarrow \mathcal{K}_x$ is injective. The second morphism is an isomorphism, so we reduce to showing that $\mathcal{O}_X(U) \rightarrow Q(U)$ is injective for all $U \subseteq X$. But if $a \in \mathcal{O}_X(U)$ and $0 = a/1 \in Q(U)$ then $sa = 0$ for some $s \in S(U)$. By definition $germ_x s$ is regular in $\mathcal{O}_{X, x}$ for all $x \in U$ and consequently $germ_x a = 0$ for all $x \in U$, which implies that $a = 0$ as required.
\end{proof}

We recall that the cokernel $\mathcal{K}^* / \cc{O}^*$ of the subobject $\cc{O}^* \rightarrow \mathcal{K}^*$ is the sheafification of the presheaf $U \mapsto \mathcal{K}^*(U) / \cc{O}^*(U)$ of abelian groups.

\begin{proposition}
    \label{proposition_1_3_1}
    Let $X$ be a scheme and $\phi : \mathcal{O}_X \rightarrow \mathcal{B}$ a morphism of sheaves of commutative rings on $X$ with the property that for every open $U \subseteq X$ the morphism $\phi_U : \mathcal{O}_X(U) \rightarrow \mathcal{B}(U)$ sends $S(U)$ to units. Then there is a unique morphism $\psi : \mathcal{K} \rightarrow \mathcal{B}$ of sheaves of rings making the following diagram commute
\begin{center}
        \begin{tikzpicture}
        \matrix(m)[matrix of math nodes,row sep=3.0em, column sep=5.5em,text height=1.5ex, text depth=0.25ex] 
        {  \cc{K} &  \cc{B}\\
            \cc{O}_{X} &  \\};
            \path[dotted, ->](m-1-1) edge node[above]{$\psi$} (m-1-2);
            \path[->](m-2-1) edge node[left]{} (m-1-1);
            \path[->](m-2-1) edge node[above]{$\phi$} (m-1-2);
 \end{tikzpicture}
\end{center}
\end{proposition}

\begin{proof}
    For nonempty open $U \subseteq X$ we obtain in the usual way a morphism of commutative rings $\psi'_U : Q(U) \rightarrow \mathcal{B}(U)$ defined by $a/s \mapsto \phi_U(a)\phi_U(s)^{-1}$ unique making the following diagram commute
\begin{center}
        \begin{tikzpicture}
        \matrix(m)[matrix of math nodes,row sep=3.0em, column sep=5.5em,text height=1.5ex, text depth=0.25ex] 
        {  Q(U) &  \cc{B}(U)\\
            \cc{O}_{X}(U) &  \\};
            \path[->](m-1-1) edge node[above]{$\psi'_U$} (m-1-2);
            \path[->](m-2-1) edge node[left]{} (m-1-1);
            \path[->](m-2-1) edge node[above]{$\phi_U$} (m-1-2);
 \end{tikzpicture}
\end{center}
This defines a morphism of presheaves of rings $\psi' : Q \rightarrow \mathcal{B}$, which induces the required morphism $\psi : \mathcal{K} \rightarrow \mathcal{B}$.
\end{proof}

\begin{lemma}
    \label{lemma_1_3_3}
    Let $X$ be a scheme and $V \subseteq X$ an open subset. Then there is a canonical isomorphism $ \mathcal{K}_V \to \mathcal{K}_X|_V$ of sheaves of algebras on $V$.

\end{lemma}

\begin{proof}
    Let $Q_X$ be the presheaf of rings on $X$ sheafifying to give $\mathcal{K}_X$ and $Q_V$ the presheaf of rings on $V$ sheafifying to give $\mathcal{K}_V$. It is clear that $Q_X|_V = Q_V$ and therefore that we have a canonical isomorphism of sheaves of rings $\mathcal{K}_X|_V \cong \mathcal{K}_V$. It is not difficult to check this is a morphism of $\mathcal{O}_X|_V$-algebras.
\end{proof}

\begin{lemma}
    \label{lemma_1_3_4}
    Let $f : X \to Y$ be an isomorphism of schemes. There is a canonical isomorphism $f^* \mathcal{K}_Y \cong \mathcal{K}_X$ of sheaves of algebras on $Y$.
\end{lemma}

\begin{proof}
    There is an obvious isomorphism of presheaves of $\mathcal{O}_Y$-algebras $Q_Y \to f_* Q_X$ which leads to an isomorphism of sheaves of algebras $\mathcal{K}_Y \cong \boldsymbol{a}(f_* Q_X) \cong f_*(\boldsymbol{a}Q_X) \cong f_* \mathcal{K}_X$, where $\boldsymbol{a}$ is the functor from the category of presheaves of $\cc{O}_X$-algebras to the category of sheaves of $\cc{O}_X$-algebras, induced by sheafification.
\end{proof}

\begin{lemma}
    \label{lemma_1_3_5}
    Let $A$ be an integral domain and set $X = \operatorname{Spec}A$. Let $S$ be the multiplicatively closed subset of all regular elements of $A$, and let $Q = S^{-1}A$. We claim there is a canonical isomorphism $\theta : \widetilde{Q} \to \mathcal{K}_X$ of sheaves of algebras on $X$.
\end{lemma}

\begin{proof}
    For the moment let $A$ be any nonzero commutative ring. By Proposition 1.3.1 it suffices to show that $\widetilde{Q}$ has the same universal property as $\mathcal{K}_X$. Let $\phi : \mathcal{O}_X \to \mathcal{B}$ be a morphism of sheaves of commutative rings on $X$ which sends $S(U)$ to units for every open $U \subset X$. We know that the functor $\widetilde{-} : \operatorname{AAlg} \to \f{Alg}(X)$ is left adjoint to the global sections functor $\Gamma(-) : \f{Alg}(X) \to \operatorname{AAlg}$. So corresponding to $\phi$ there is a morphism of algebras $\Phi : Q \to \Gamma(\mathcal{B})$ which sends elements of $S$ to units. It is not hard to see that the ring isomorphism $A \cong \Gamma(\cc{O}_X)$ identifies $S$ with $S(X)$, since if $A$ is a nonzero ring then $a \in A$ is regular in $A$ iff $a/1$ is regular in $A_p$ for every prime ideal $\f{p}$ of $A$. Therefore there is a unique morphism of rings $\Psi : Q \to \Gamma(\mathcal{B})$ with $A \to Q \to \Gamma(\mathcal{B}) = \Phi$. Using the properties of the adjunction, it is not difficult to check that the corresponding morphism of sheaves of rings $\psi : \widetilde{Q} \to \mathcal{B}$ is unique making the following diagram commute
\begin{center}
        \begin{tikzpicture}
        \matrix(m)[matrix of math nodes,row sep=3.0em, column sep=5.5em,text height=1.5ex, text depth=0.25ex] 
        {  \widetilde{Q} &  \cc{B}\\
            \cc{O}_{X} &  \\};
            \path[->](m-1-1) edge node[above]{$\psi$} (m-1-2);
            \path[->](m-2-1) edge node[left]{} (m-1-1);
            \path[->](m-2-1) edge node[above]{$\phi$} (m-1-2);
 \end{tikzpicture}
\end{center}
Next we show that the morphism of sheaves of rings $\mathcal{O}_X \to \widetilde{Q}$ sends elements of $S(U)$ to units for every open $U \subset X$. It suffices to show that $\mathcal{O}_X(D(f)) \to \widetilde{Q}(D(f))$ sends $S(D(f))$ to units for every nonzero $f \in A$. But by the same argument as above, the ring isomorphism $\mathcal{O}_X(D(f)) \cong A_f$ identifies $S(D(f))$ with the set of regular elements of the ring $A_f$, so we have only to show that $A_f \to Q_f$ sends regular elements to units.

Now assume that $A$ is an integral domain. If $a / f^m$ is regular in $A_f$ it follows that $a$ is nonzero in $A_f$, and hence regular in $A$, so clearly $a / f^m$ is a unit in $Q_f$, as required. Using the universal property we obtain a \emph{unique} isomorphism $\theta : \widetilde{Q} \to \mathcal{K}_X$ of sheaves of algebras on $X$.
\end{proof}

\begin{proposition}
    \label{proposition_1_3_2}
    If $X$ is an integral scheme then $\mathcal{K}_X$ is a quasi-coherent sheaf of $\mathcal{O}_X$-algebras.
\end{proposition}

\begin{proof}
    Combining Lemma 1.3.3 and Lemma 1.3.4 we reduce to the case $X = \operatorname{Spec} A$ for an integral domain $A$, which is handled in Lemma 1.3.5.
\end{proof}

\begin{definition}
    \label{definition_1_3_2}
    A \emph{Cartier divisor} on a scheme $X$ is a global section of the sheaf $\mathcal{K}^* / \mathcal{O}^*$. A Cartier divisor is \hl{principal} if it is in the image of the natural map $\mathcal{K}^*(X) \to \mathcal{K}^*(X)/\mathcal{O}^*(X)$. Two Cartier divisors are \hl{linearly equivalent} if their difference is principal. Although the group operation on $\mathcal{K}^* / \mathcal{O}^*$ is multiplication, we will use the language of additive groups when speaking of Cartier divisors, so as to preserve the analogy with Weil divisors.
\end{definition}

\begin{remark}
    \label{remark_1_3_1}
    If $X$ is the zero scheme, then the group of Cartier divisors is the zero group. Otherwise, giving a Cartier divisor on $X$ is equivalent to giving a cover $\{U_i\}_{i \in I}$ of $X$ by affine open sets, and for each $i \in I$ an element $f_i \in \mathcal{K}^*(U_i)$ such that for all $i, j \in I$, $f_i|_{U_i \cap U_j} / f_j|_{U_i \cap U_j}$ belongs to the image of $\mathcal{O}^*(U_i \cap U_j)$ in $\mathcal{K}^*(U_i \cap U_j)$. We represent this situation by saying that \hl{the divisor is represented by the functions} $\{(U_i, f_i)\}_{i \in I}$. 
\end{remark}

\begin{lemma}
    If $X$ is an integral scheme then $\mathcal{K}$ is isomorphic as a sheaf of $\mathcal{O}_X$-algebras to the constant sheaf associated to the function field $K$ of $X$.
\end{lemma}

\begin{proof}
    We begin by proving that for $x \in X$ there is an isomorphism of rings $Q_x \cong K$. First notice that for any nonempty open subset $U$ the set $S(U)$ consists precisely of those $s \in \mathcal{O}_X(U)$ with $s_\xi \neq 0$ (where $\xi$ is the generic point of $X$). So the ring morphism $\mathcal{O}_X(U) \to \mathcal{O}_{X,\xi} = K$ sends the elements of $S(U)$ to units and induces a morphism of rings $Q(U) \to K$, and therefore a morphism of rings $\rho_x : Q_x \to K$ for any $x \in K$, defined by $(U, a/s) \mapsto a_\xi(s_\xi)^{-1}$.

To show $\rho_x$ injective for all $x$, it suffices to show that $Q(U) \to K$ is injective for nonempty open $U$. But if $a, b \in \mathcal{O}_X(U)$ and $s, t \in S(U)$ are such that $a s_\xi = b t_\xi \Longrightarrow t_\xi^{-1}(at - bs) = 0$ and thus $at - bs = 0$ in $\mathcal{O}_X(U)$, showing that $a/s = b/t \in Q(U)$. The map $\rho_x$ is surjective since $K$ is the quotient field of $\mathcal{O}_{X,\xi}$ for any $x \in K$. Hence $\rho_x$ is an isomorphism of rings.

For nonempty open $U$ define $\mathcal{K}(U) \to K$ by $\alpha \mapsto \rho_\xi(\alpha(\xi))$. Elements of $\mathcal{K}(U)$ map points $x \in U$ to germs $a(x) \in Q_x$, and since $X$ is integral $\rho_\xi(a(x))$ will be constant for all $x \in U$. So we may as well use $\xi$. It is not difficult to see that this gives an isomorphism of sheaves of rings $\mathcal{K}$ with the constant sheaf on $X$ corresponding to the field $K$. Clearly the sheaf of abelian groups $\mathcal{K}^*$ is isomorphic to the constant sheaf on $X$ corresponding to the multiplicative abelian group $K^*$.
\end{proof}

Notice that the isomorphism of abelian groups $\mathcal{K}^*(U) \cong K^*$ for nonempty $U$ fits into the following commutative diagram:
\begin{center}
        \begin{tikzpicture}
        \matrix(m)[matrix of math nodes,row sep=3.0em, column sep=5.5em,text height=1.5ex, text depth=0.25ex] 
        {  \cc{O}^*(U) &  \cc{K}^*(U)\\
             &  K^*\\};
            \path[->](m-1-1) edge node[above]{} (m-1-2);
            \path[->](m-1-1) edge node[left]{} (m-2-2);
            \draw[double, ->](m-1-2) -- (m-2-2);
 \end{tikzpicture}
\end{center}

\begin{lemma}
    Set $X = \mathbb{P}^n_k$ for a field $k$ and fix a nonzero homogeneous polynomial $f \in S_1$. Identify the function field K with $S_{((0))}$. For $x \in X$ let $l$ be an arbitrary integer with $x \in D_+(x_i)$ and define
\[
C_f(x) = (X, f/x_i + \mathcal{O}^*(X))
\]
Then $C_f$ is a Cartier divisor. Equivalently $C_f$ is defined by the family of sections $\{(D_+(x_i), f/x_i + \mathcal{O}^*)\}_{0 \leq i \leq n}$. Given $0 \leq \ell \leq n$ we write $C_\ell$ for the Cartier divisor $C_{x_\ell}$.
\end{lemma}

\begin{proof}
    First we have to check the definition makes sense. Suppose we have $x \in D_+(x_i) \cap D_+(x_j)$. To show the germs $(X, f/x_i + \mathcal{O}^*(X))$, $(X, f/x_j + \mathcal{O}^*(X))$ agree, it suffices to show that $(f/x_i)/(f/x_j) \in S_{((0))}$ corresponds to an element $\mathcal{O}^*(U) \subseteq \mathcal{K}^*(U)$ for some open neighbourhood $U$ of $x$. If both we take $U = D_+(x_i) \cap D_+(x_j)$ then $x_j / x_i \in \mathcal{O}^*(U)$, as required.

Next we have to check that $C_\ell \in \Gamma(X, \mathcal{K}^*/\mathcal{O}^*)$. But for every $y \in D_+(x_i)$ we have $C_\ell(y) = (D_+(x_i), x_\ell/x_i + \Gamma(D_{+}(x_i), \mathcal{O}^*))$, so this holds.
\end{proof}

\begin{lemma}
    Let $X$ be a normal scheme satisfying $(*)$ and let $U$ be a nonempty open subset. If $f, g \in K^*$ are such that $v_Y(f) = v_Y(g)$ for all prime divisors $Y$ with $Y \cap U \neq \emptyset$, then $f/g \in \mathcal{O}^*(U)$.
\end{lemma}

\begin{proof}
    Considering $\mathcal{O}_X(U)$ as a subring of $K$, it suffices to show that $f/g, g/f \in \mathcal{O}_X(U)$. By symmetry it suffices to show $f/g \in \mathcal{O}_X(U)$, and it suffices to produce an open cover $U = \bigcup U_i$ of $U$ with $f/g \in \mathcal{O}_X(U_i)$ for all $i$. Given $x \in U$ let $V \cong \operatorname{Spec} A$ be an affine open neighborhood of $x$. Since $X$ is normal, $A$ is a normal noetherian domain. Let $\mathfrak{p}$ be a prime ideal of height 1 in $A$, then $V(\mathfrak{p})$ corresponds to a prime divisor of $U$, which is the restriction of a prime divisor of $Y$ of $X$ (with generic point $\eta \in U$ corresponding to $\mathfrak{p}$). Let $Q$ be the quotient field of $A$, and let $h \in Q$ be the image of $f/g$ under the isomorphism $K \cong Q$. Then $h \in A_{\f{p}}$ since $v_Y(f/g) = 0$ implies $f/g \in \mathcal{O}_{X,\eta}$. By Proposition 1.1.1 it follows that $h \in A \subseteq K$ and hence $f/g \in \mathcal{O}_X(V)$, which gives the required open cover of $U$ and completes the proof.
\end{proof} 

As a particular case this shows that if $X$ is a normal scheme satisfying $(*)$ and $U$ is a nonempty open subset, if $f \in K^*$ and $v_Y(f) = 0$ for all prime divisors $Y$ with $Y \cap U \neq \emptyset$ then $f \in \mathcal{O}^*(U)$.

\begin{definition}
    A scheme $X$ is \hl{locally factorial} if for all $x \in X$ the local ring $\mathcal{O}_{X,x}$ is a unique factorization domain. Since a regular local ring is a unique factorisation domain, any nonsingular scheme is locally factorial.
\end{definition}

\begin{proposition}
    Let $X$ be an integral, separated noetherian scheme which is locally factorial. Then the group $\mathrm{Div}X$ of Weil divisors on $X$ is isomorphic to the group of Cartier divisors $\Gamma(X, \mathcal{K}^*/\mathcal{O}^*)$, and furthermore, the principal Weil divisors correspond to the principal Cartier divisors under this isomorphism.
\end{proposition}

\begin{proof}
    First note that $X$ is normal, hence satisfies $(*)$, whence a UFD is normal. So it makes sense to talk about Weil divisors. In the case where $X$ has no prime divisors, then $X = \operatorname{Spec} k$ for a field $k$ and the presheaf of abelian groups $U \mapsto \mathcal{K}^*(U)/\mathcal{O}^*(U)$ is the zero presheaf, so $\mathrm{Div}X = \Gamma(X, \mathcal{K}^*/\mathcal{O}^*) = 0$, and this isomorphism clearly preserves principal divisors. So we can assume that $X$ has at least one principal divisor.

Let a Cartier divisor $C$ be given, and let $Y$ be a prime divisor of $X$ with generic point $\eta$. Let $C(\eta) = (U, f + \cc{O}^*(U)$ for some open neighbourhood $U$ of $\eta$ and $f \in \mathcal{K}^*(U)$. Take the coefficient $C_Y$ to by the integer $v_Y(f)$, where we use the isomorphism of groups $\mathcal{K}^*(U) \cong K^*$ to identify $f$ with an element of $K^*$. To be precise, this integer is $v_Y(\rho_\eta (f(\eta)))$.This independent of the choice of element $f$ used to represent the germ $C(\eta)$, since if $g \in \mathcal{K}^*(V)$ gives the same germ then there is an open neighborhood $\eta \in W \subseteq U \cap V$ with $f|_W/g|_W \in \mathcal{O}^*(W) \subseteq \mathcal{K}^*(W)$. For any element $a \in \mathcal{O}_X(W)$ the germ $a_\xi \in K^*$ is assigned a non-negative value by $v_Y$ since it belongs to the domain of definition of $a_\xi$. And by assumption $f|_W/g|_W$ is the image in $\mathcal{K}^*(W)$ of a unit in $\mathcal{O}_X(W)$, whose value under $v_Y$ must be zero. Consequently $v_Y(f/g) = v_Y(f) - v_Y(g) = 0$, so $v_Y(f) = v_Y(g)$.

We claim that $C_Y \neq 0$ for only finitely many prime divisors $Y$. Since $X$ is noetherian we can cover it in a finite number of nonempty open sets $U_i$ together with $f_i \in \mathcal{K}^*(U_i)$ such that for all $y \in U_i, C(y) = (U_i, f_i + \cc{O}^*(U_i))$. By Lemma 1.1.3, $v_Y(f_i) \neq 0$ for only finitely many $Y$, so it follows that $C_Y \neq 0$ for only finitely many $Y$. Thus we obtain a well-defined Weil divisor $D = \sum C_Y \cdot Y$ on $X$.

Conversely, if $D$ is a Weil divisor on $X$, let $x \in X$ be any point other than the generic point $\xi$.
By Lemma 1.1.2 there is at least one prime divisor passing through $x$, and if $f : \operatorname{Spec}(\mathcal{O}_{X, x}) \to X$ is canonical then $Y \mapsto f^{-1}Y$ gives an injective map from the set of the divisors passing through $x$ to the prime divisors of $\operatorname{Spec}(\mathcal{O}_{X, x})$. So $D$ induces a Weil divisor $D_x$ on $\operatorname{Spec}(\mathcal{O}_{X, x})$. The divisor $D_x$ is principal by Proposition 1.1.2, so $D_x = (f_x)$ for some $f_x \in Q^*$, where $Q$ is the function field of $\operatorname{Spec}(\mathcal{O}_{X, x})$. Let $h_{x} \in K^*$ be the corresponding element under the isomorphism $K \cong Q$. Then the principal divisor $(h_x)$ has the same values as $D$ on any prime divisor meeting $x$. Since there are only finitely many prime divisors which do not contain $x$ on which either $D$ or $(h_x)$ has a nonzero value, there is an open neighbourhood $U_x$ of $x$ such that $D$ and $(h_x)$ have the same restriction to $U_x$. It follows from Lemma 1.3.8 that
\[
h_x / h_y \in \mathcal{O}^*(U_x \cap U_y)
\]
for any $x, y \in X$. So the elements of $\mathcal{K}^*(U_x)$ corresponding to the $h_x$ patch together to give a Cartier divisor on $X$.

It is clear that any Weil divisor $D$ and corresponding Cartier divisor $C$ that the Weil divisor produced $\sum_Y C_Y \cdot Y$ produced from $C$ is just $D$. In particular the construction in the previous paragraph is independent of the chosen $f_x \in Q^*$ with $(f_x) = D$ (which was the only choice involved). In the other direction, given a Cartier divisor $C$ and corresponding Weil divisor $D = \sum_Y C_Y \cdot Y$ pick for each $x \in X$ an open set $U_x$ and $h_x \in \mathcal{K}^*(U_x)$ such that $C(y) = (U_x, h_x + \mathcal{O}^*(U_x))$ for all $y \in U_x$. Let $f_x \in Q^*$ correspond to $h_x$. Then $(f_x) = D_x$ and $C$ is the Cartier divisor constructed from $D$, as required.

We have established a bijection $\operatorname{Div} X \to \Gamma(X, \mathcal{K}^*/\mathcal{O}^*)$, which is easily seen to be a morphism of abelian groups. If $f \in \mathcal{K}^*(X)$ then the Weil divisor corresponding to $f$ is just the principal divisor $(f)$ (identifying $\mathcal{K}^*(X)$ with $K^*$), so the isomorphism identifies principal Weil divisors with principal Cartier divisors.
\end{proof}

\begin{definition}
    Let $X$ be a scheme. The \hl{Cartier divisor class group} of $X$, denoted $\mathrm{CaCl} X$, is the group of Cartier divisors modulo the principal Cartier divisors.
\end{definition}

Proposition 1.3.3 implies that for an integral, separated noetherian locally factorial scheme $X$ we have isomorphisms of abelian groups
\[
\operatorname{Div} X \cong \Gamma(X, \mathcal{K}^*/\mathcal{O}^*) \text{ and } \operatorname{Cl} X \cong \operatorname{CaCl} X.
\]
In particular this is true when $X$ is a nonsingular variety over a field $k$.

\begin{remark}
    Let $X = \mathbb{P}^n_k$ where $k$ is a field and $n \geq 1$. We have already observed that $X$ is a variety over $k$. In fact, it is a nonsingular variety since it is covered by affine open subsets isomorphic to $\operatorname{Spec} k[x_1, \dotsc, x_n]$, and the local rings of $k[x_1, \dotsc, x_n]$ are all regular. Since a regular local ring is a UFD, $X$ is locally factorial and we can apply Proposition 1.3.3.
\end{remark}

\begin{lemma}
    Let $X = \mathbb{P}^n_k$ where $k$ is a field and $n \geq 1$. Under the isomorphism
$\operatorname{Div} X \cong \Gamma(X, \mathcal{K}^*/\mathcal{O}^*)$ the hyperplane $V(g)$ corresponds to the Cartier divisor $C_g$ for any nonzero homogeneous polynomial $g \in S_1$.
\end{lemma}

\begin{proof}
    See Lemma 1.3.7 for the definition of the Cartier divisor $C_g$. Let $Y = V(f)$ be a prime divisor of $X$ with generic point $\eta$. Then $C_g(\eta) = (X, g/f_i + \mathcal{O}^*(X))$ where $\eta \subset D_+(f_i)$. In any case, $f \neq z_i$ and so clearly $v_g(z_i) = 0$, unless $f$ is an associate of $g$ in which case $C_g(\eta) = 1$. So Proposition 1.3.3 associates $C_g$ with $V(g)$.
\end{proof}

\section{Invertible Sheaves}

We recall that an \hl{invertible sheaf} on a ringed space $X$ is a locally free $\cc{O}_X$-module of rank 1. We will see now that invertible sheaves on a scheme are closely related to divisor classes modulo linear equivalence.

\begin{proposition}
    If $\mathcal{L}$ and $\mathcal{M}$ are invertible sheaves on a ringed space $X$, so is $\mathcal{L}\otimes \mathcal{M}$. If $\mathcal{L}$ is any invertible sheaf on $X$, then there exists an invertible sheaf $\mathcal{L}^{-1}$ on $X$ such that $\mathcal{L}\otimes \mathcal{L}^{-1} \cong \mathcal{O}_X$.
\end{proposition}


\begin{proof}
The first statement is clear, since $\mathcal{L}$ and $\mathcal{M}$ are both locally free of rank $1$, and $\mathcal{O}_X \otimes \mathcal{O}_Y \cong \mathcal{O}_Y$. For the second statement, let $\mathcal{L}$ be any invertible sheaf, and take $\mathcal{L}^{-1}$ to be the dual sheaf $\mathcal{L}^{\vee} = \mathcal{H}om(\mathcal{L}, \mathcal{O}_X)$. Then
\[
\mathcal{L}^{\vee} \otimes \mathcal{L} \cong \mathcal{H}om(\mathcal{L}, \mathcal{L}) = \mathcal{O}_X
\]
by (\cite{one}, II, Ex.\ 5.1).
\end{proof}

\begin{definition}
     Let $D$ be a Cartier divisor on a scheme $X$, represented by $\{(U_i, f_i)\}$ as above. Let $\mathcal{L}(D)$ be the subsheaf of the $\mathcal{O}_X$-module $\mathcal{K}_X^*$ generated by the set $\{f_i^{-1}\}$. We call $\mathcal{L}(D)$ the sheaf associated to $D$. Clearly $\mathcal{L}(0) = \mathcal{O}_X$.
\end{definition}

\begin{lemma}
    Let $D$ be a Cartier divisor on a scheme $X$. The $\mathcal{O}_X$-module $\mathcal{L}(D)$ is independent of the matching family $\{(U_i, f_i)\}$ chosen to represent $D$, and for $U \subseteq U_i$, $\mathcal{L}(D)$ is the $\mathcal{O}_X(U)$-submodule generated by $f_i^{-1}|_U$.
\end{lemma}

\begin{proof}
    Let $D$ be a Cartier divisor represented by $\{(U_i, f_i)\}$. Then for $x \in U_i$ the submodule $G_x$ generated by the set $\{\text{germ}_x(f_i^{-1}) \mid x \in U_i\}$ is in fact $G_x = (\text{germ}_x f_i^{-1})$. To see this, note that by definition $f_i|_{U_i \cap U_j}/f_j|_{U_i \cap U_j} \in \mathcal{O}^*(U_i \cap U_j)$ so in the $\mathcal{O}_X(U_i \cap U_j)$-module $\mathcal{K}(U_i \cap U_j)$ we have $f_i^{-1}|_{U_i \cap U_j} = u f_j^{-1}|_{U_i \cap U_j}$ for a unit $u$. So for all $x \in U_i \cap U_j$, we have $\text{germ}_x f_j^{-1} \in (\text{germ}_x f_i^{-1})$. This proves that $G_x = (\text{germ}_x f_i^{-1})$ for $x \in U_i$.

If $\{(V_j, g_j)\}$ is another cover amalgamating to give $D$ and $x \in X$ then say $x \in U_i \cap V_j$. Then in $Q_x$ we have $(V_j, g_j + \mathcal{O}^*(V_j)) = D(x) = (U_i, f_i + \mathcal{O}_X(U_i))$. As before this shows that $\text{germ}_x g_j^{-1}$ and $\text{germ}_x f_i^{-1}$ generate the same submodule of $\mathcal{K}_x$, and so by the previous paragraph the two covers determine the same subsheaf $\mathcal{L}(D)$ of $\mathcal{K}_x$.

For the second claim suppose $a \in \Gamma(U, \mathcal{L}(D))$ for a nonempty open subset $U$. The fact that $\text{germ}_x a \in (\text{germ}_x f_i^{-1})$ for all $x \in U$ means we can find a cover of $U$ by nonempty open $W_\alpha$ with $r_\alpha \in \mathcal{O}_X(W_\alpha)$ such that $a|W_\alpha = r_\alpha \cdot f_i^{-1}|_{W_\alpha}$. If $\varphi : \mathcal{O}_X \to \mathcal{K}$ is the canonical monomorphism of sheaves of rings (which gives $\mathcal{K}$ its $\mathcal{O}_X$-module structure) then $r_\alpha \cdot f_i^{-1}|_{W_\alpha} = \varphi_{W_\alpha}(r_\alpha) f_i^{-1}|_{W_\alpha}$. Since $f_i$ is a unit and $\varphi$ is a monomorphism it follows that the $r_\alpha$ form a matching family, which yields $r \in \mathcal{O}_X(U)$ with $r|_{W_\alpha} = r_\alpha$ for all $\alpha$. Clearly $a = r \cdot f_i^{-1}|_U$, as required.
\end{proof}

Before giving the next result we need to study more closely which sections of $\mathcal{K}$ are units.

\begin{lemma}
    Let $X$ be a scheme and $U \subseteq X$ an open subset. Then $f \in \mathcal{K}(U)$ is a unit if and only if $\operatorname{germ}_x f \in \mathcal{K}_x$ is $\mathcal{O}_{X,x}$-torsion-free for all $x \in U$.
\end{lemma}

\begin{proof}
    This is trivial for $U = \varnothing$, so we may assume $U$ is nonempty. Suppose $f(x) \in Q_x$ is a unit for all $x \in U$ and define $g(x) = f(x)^{-1}$. Then $g$ is a well-defined element of $\mathcal{K}(U)$ since if $x \in U$ is given we can find $x \in V \subseteq U$ and $a \in Q(V)$ with $f(y) = (V,a)$ for all $y \in V$. If $(W, b) = (V,a)^{-1}$ in $Q_x$ then $(W,b) = (V,a)^{-1}$ in $Q_y$ for all $y$ in some open $x \in Q \subseteq V \cap W$ and consequently $g(y) = (Q, b|_Q)$ for all $y \in Q$. So $f$ is a unit in $\mathcal{K}(U)$ iff $f(x)$ is a unit in $Q_x$ for all $x \in U$.

Since there is an isomorphism of rings $Q_x \cong \mathcal{K}_x$, we have reduced to showing that $f(x)$ is a unit in $Q_x$ for all $x \in U$ iff $f(x)$ is $\mathcal{O}_{X,x}$-torsion-free for all $x \in U$. One implication is clear. For the other, suppose $x \in U$ is given, and let $x \in V \subseteq U$ be such that $f(y) = (V, a/s) \in Q_y$ for all $y \in V$, where $a \in \mathcal{O}_X(V)$ and $s \in S(V)$. By assumption $(V, a/s)$ is torsion-free in $Q_y$ for all $y \in V$, from which it follows that $(V, a)$ is torsion-free and therefore regular in $\mathcal{O}_{X,y}$ for all $y \in V$. That is, $a \in S(V)$, from which it follows that $(V, a/s)$ is a unit in $Q_x$. So $f(x)$ is a unit for all $x \in U$, as required.
\end{proof}
\begin{cor}
    Let $X$ be a scheme and $\mathcal{L}$ an invertible submodule of $\mathcal{K}$. Then for any $x \in X$, $\mathcal{L}_x$ is generated as an $\mathcal{O}_{X,x}$-submodule of $\mathcal{K}_x$ by a unit.
\end{cor}

\begin{proof}
    Omitted.
\end{proof}

\begin{lemma}
    Let $A$ be a commutative ring and $B$ a commutative $A$-algebra. Let $M$, $N$ be $A$-submodules of $B$ with one of $M$, $N$ generated as an $A$-module by a regular element of $B$. Then there is a canonical isomorphism of $A$-modules $M \otimes_A N \to MN$ given by $m \otimes n \mapsto mn$.
\end{lemma}

\begin{proof}
    We assume without loss of generality that $M$ is generated as an $A$-module by a regular element $a \in B$. The map $M \times N \to M \cdot N$ given by $(m,n) \mapsto mn$ is clearly $A$-bilinear, and induces a surjective morphism of $A$-modules $M \otimes_A N \to M \cdot N$. Suppose that $\sum_i (a_i \otimes b_i)$ is mapped to zero in $B$. Then $(\sum_i a_i b_i) = 0$ and therefore $\sum_i a_i b_i = 0$. But then
\[
\sum_i (a_i \otimes b_i) = a \otimes (\sum_i a_i b_i) = 0
\]
therefore $M \otimes_A N \to M \cdot N$ is an isomorphism.
\end{proof}

\begin{definition}
    Let $X$ be a scheme and $\mathcal{F}$, $\mathcal{G}$ submodules of the commutative $\mathcal{O}_X$-algebra $\mathcal{K}$. We already know the product $\mathcal{F} \cdot \mathcal{G}$, which is a submodule of $\mathcal{K}$. If we identify $\mathcal{O}_X$ with a submodule of $\mathcal{K}$, then it is clear that $\mathcal{O}_X \cdot \mathcal{F} = \mathcal{F}$ for any submodule $\mathcal{F}$ of $\mathcal{K}$.
\end{definition}

\begin{proposition}
    Let $X$ be a scheme and $\mathcal{L}$, $\mathcal{M}$ submodules of $\mathcal{K}$ with one of $\mathcal{L}$, $\mathcal{M}$ invertible. Then there is a canonical isomorphism $\mathcal{M} \otimes \mathcal{L} \to \mathcal{M} \cdot \mathcal{L}$ of sheaves of modules on $X$.
\end{proposition}
\begin{proof}
    By definition we have an epimorphism of sheaves of modules $\mathcal{M} \otimes \mathcal{L} \to \mathcal{M} \cdot \mathcal{L}$, so it suffices to show that the map $\mathcal{M}_x \otimes_{\mathcal{O}_{X,x}} \mathcal{L}_x \to \mathcal{K}_x$ is injective for every $x \in X$. This follows immediately from Lemma 1.4.3 and Corollary 1.4.1.
\end{proof} 

\begin{proposition}
    Let $X$ be a scheme. Then:
\begin{itemize}
    \item[(a)] for any Cartier divisor $D$, $\mathcal{L}(D)$ is an invertible sheaf on $X$. The map $D \mapsto \mathcal{L}(D)$ gives a $1$-$1$ correspondence between Cartier divisors on $X$ and invertible subsheaves of $\mathcal{K}$;
    \item[(b)] $\mathcal{L}(D_1 - D_2) \cong \mathcal{L}(D_1) \otimes \mathcal{L}(D_2)^{-1}$;
    \item[(c)] $D_1 = D_2$ if and only if $\mathcal{L}(D_1) \cong \mathcal{L}(D_2)$ as abstract invertible sheaves (i.e., disregarding the embedding in $\mathcal{K}$).
\end{itemize}
\end{proposition}
\begin{proof}
    (a) Since each $f_i \in \Gamma(U_i, \mathcal{K}^*)$, the map $\mathcal{O}_{U_i} \to \mathcal{L}(D)|_{U_i}$ defined by $1 \mapsto f_i^{-1}$ is an isomorphism. Thus $\mathcal{L}(D)$ is an invertible sheaf. The Cartier divisor $D$ can be recovered from $\mathcal{L}(D)$ together with its embedding in $\mathcal{K}$, by taking $f_i$ on $U_i$ to be the inverse of a local generator of $\mathcal{L}(D)$. For any invertible subsheaf of $\mathcal{K}$, this construction gives a Cartier divisor, so we have a $1$-$1$ correspondence as claimed.

(b) If $D_1$ is locally defined by $f_i$ and $D_2$ is locally defined by $g_i$, then $\mathcal{L}(D_1 - D_2)$ is locally generated by $f_i^{-1}g_i$, so $\mathcal{L}(D_1 - D_2) = \mathcal{L}(D_1) \cdot \mathcal{L}(D_2)^{-1}$ as subsheaves of $\mathcal{K}$. This product is clearly isomorphic to the abstract tensor product $\mathcal{L}(D_1) \otimes \mathcal{L}(D_2)^{-1}$.

(c) Using (b), it will be sufficient to show that $D = D_1 - D_2$ is principal if and only if $\mathcal{L}(D) \cong \mathcal{O}_X$. If $D$ is principal, defined by $f \in \Gamma(X, \mathcal{K}^*)$, then $\mathcal{L}(D)$ is globally generated by $f^{-1}$, so sending $1 \mapsto f^{-1}$ gives an isomorphism $\mathcal{O}_X \cong \mathcal{L}(D)$. Conversely, given such an isomorphism, the image of $1$ gives an element of $\Gamma(X, \mathcal{K}^*)$ whose inverse will define $D$ as a principal divisor.
\end{proof}

\begin{cor}
    Let $X$ be a scheme. Then
\begin{enumerate}
    \item[(i)] The map $D \mapsto \mathcal{L}(D)$ induces an isomorphism of abelian groups between $\operatorname{CaCl} X$ and $\operatorname{Inv} \mathcal{K} / P$, where $P$ denotes the subgroup of principal invertible submodules of $\mathcal{K}$.
    \item[(ii)] The map $D \mapsto \mathcal{L}(D)$ gives an injective morphism of abelian groups $\operatorname{CaCl} X \longrightarrow \operatorname{Pic} X$.
\end{enumerate}
\end{cor}

\begin{proof}
    Omitted.
\end{proof}

\begin{remark}
    The map $\operatorname{CaCl} X \rightarrow \operatorname{Pic} X$ may not be surjective, because there may be invertible sheaves on $X$ which are not isomorphic to any invertible subsheaf of $\mathcal{K}$. But it can be proved to be surjective under special circumstances.
\end{remark}

\begin{lemma}
    Let $X$ be an integral scheme and $\mathcal{L}$ an invertible sheaf. If $\xi$ is the generic point then the canonical morphism of abelian groups $\mathcal{L}_x \longrightarrow \mathcal{L}_\xi$ is injective for all $x \in X$. Consequently $\mathcal{L}(U) \longrightarrow \mathcal{L}_\xi$ is injective for any open $U \subset X$.
\end{lemma}

\begin{proof}
    The morphism is defined by $(U, a) \mapsto (U, a)$. Find an open neighborhood $U$ of $x$ (which must contain $\xi$) with $\mathcal{L}|_U \cong \mathcal{O}_X|_U$ and reduce to the case where $\mathcal{L} = \mathcal{O}_X$, which is easily checked.
\end{proof}

\begin{proposition}
    If $X$ is an integral scheme, the morphism $\operatorname{CaCl} X \rightarrow \operatorname{Pic} X$ is an isomorphism.
\end{proposition}

\begin{proof}
    We have only to show that every invertible sheaf is isomorphic to a submodule of $\mathcal{K}$, which in this case is isomorphic to the constant sheaf $K$, where $K$ is the function field of $X$. So given an invertible sheaf $\mathcal{L}$ it suffices to show there is a monomorphism $\psi: \mathcal{L} \longrightarrow K$. We choose an isomorphism $\mathcal{L}_\xi \cong K$ and define $\psi$ to be the composite $\mathcal{L}(U) \longrightarrow \mathcal{L}_\xi \cong K$. This is easily checked to be a monomorphism of $\mathcal{O}_X$-modules, which completes the proof. In particular this result shows that if $X$ is integral, $\operatorname{Pic} X$ is small (that is, bijective to an element of our universe).
\end{proof}

\begin{cor}
    If $X$ is a noetherian, integral, separated, locally factorial scheme, then there is a canonical isomorphism of abelian groups $\operatorname{Cl} X \cong \operatorname{Pic} X$.
\end{cor}

\begin{proof}
    This follows from Proposition 1.4.4 and Proposition 1.3.3.
\end{proof}

\begin{example}
    If $k$ is an algebraically closed field and $X$ is a nonsingular variety over $k$, then since a regular local ring is a UFD, $X$ satisfies the conditions of the Corollary 1.4.3 and $\operatorname{Cl} X \cong \operatorname{Pic} X$.
\end{example}